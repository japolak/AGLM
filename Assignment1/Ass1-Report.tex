%% LyX 2.2.1 created this file.  For more info, see http://www.lyx.org/.
%% Do not edit unless you  know what you are doing.
\documentclass{article}\usepackage[]{graphicx}\usepackage[]{color}
% maxwidth is the original width if it is less than linewidth
% otherwise use linewidth (to make sure the graphics do not exceed the margin)
\makeatletter
\def\maxwidth{ %
  \ifdim\Gin@nat@width>\linewidth
    \linewidth
  \else
    \Gin@nat@width
  \fi
}
\makeatother

\definecolor{fgcolor}{rgb}{0.345, 0.345, 0.345}
\newcommand{\hlnum}[1]{\textcolor[rgb]{0.686,0.059,0.569}{#1}}%
\newcommand{\hlstr}[1]{\textcolor[rgb]{0.192,0.494,0.8}{#1}}%
\newcommand{\hlcom}[1]{\textcolor[rgb]{0.678,0.584,0.686}{\textit{#1}}}%
\newcommand{\hlopt}[1]{\textcolor[rgb]{0,0,0}{#1}}%
\newcommand{\hlstd}[1]{\textcolor[rgb]{0.345,0.345,0.345}{#1}}%
\newcommand{\hlkwa}[1]{\textcolor[rgb]{0.161,0.373,0.58}{\textbf{#1}}}%
\newcommand{\hlkwb}[1]{\textcolor[rgb]{0.69,0.353,0.396}{#1}}%
\newcommand{\hlkwc}[1]{\textcolor[rgb]{0.333,0.667,0.333}{#1}}%
\newcommand{\hlkwd}[1]{\textcolor[rgb]{0.737,0.353,0.396}{\textbf{#1}}}%
\let\hlipl\hlkwb

\usepackage{framed}
\makeatletter
\newenvironment{kframe}{%
 \def\at@end@of@kframe{}%
 \ifinner\ifhmode%
  \def\at@end@of@kframe{\end{minipage}}%
  \begin{minipage}{\columnwidth}%
 \fi\fi%
 \def\FrameCommand##1{\hskip\@totalleftmargin \hskip-\fboxsep
 \colorbox{shadecolor}{##1}\hskip-\fboxsep
     % There is no \\@totalrightmargin, so:
     \hskip-\linewidth \hskip-\@totalleftmargin \hskip\columnwidth}%
 \MakeFramed {\advance\hsize-\width
   \@totalleftmargin\z@ \linewidth\hsize
   \@setminipage}}%
 {\par\unskip\endMakeFramed%
 \at@end@of@kframe}
\makeatother

\definecolor{shadecolor}{rgb}{.97, .97, .97}
\definecolor{messagecolor}{rgb}{0, 0, 0}
\definecolor{warningcolor}{rgb}{1, 0, 1}
\definecolor{errorcolor}{rgb}{1, 0, 0}
\newenvironment{knitrout}{}{} % an empty environment to be redefined in TeX

\usepackage{alltt}
\usepackage[sc]{mathpazo}
\usepackage[T1]{fontenc}
\usepackage[english]{babel}
\usepackage[utf8]{inputenc}
\usepackage{geometry}
\usepackage{dsfont}
\usepackage{indentfirst}
\usepackage{fancyhdr}
\usepackage{amsmath}

% Margins of the document
\geometry{verbose,tmargin=2.5cm,bmargin=2.5cm,lmargin=2.5cm,rmargin=2.5cm}

% Header and footer for all pages
\pagestyle{fancy}
\fancyhf{}
\renewcommand{\headrulewidth}{0pt}
\rfoot{Page \thepage}

% Header and footer for first page
\fancypagestyle{plain}{%
  \renewcommand{\headrulewidth}{0pt}%
  \fancyhf{}%
  \rhead{ETH Zurich}
  \lhead{Applied Generalized Linear Models \\ Spring Semester 2020}
  \rfoot{Page \thepage}
}

%% Global Settings



\DeclareMathOperator{\SSR}{SSR}
\DeclareMathOperator{\SSreg}{SSreg}
\DeclareMathOperator{\SST}{SST}


%% ---------- BEGIN DOCUMENT -----------------------------------------------
\IfFileExists{upquote.sty}{\usepackage{upquote}}{}
\begin{document}

%% Tiltle
\title{Assignment 1}
\author{Milan Kuzmanovic, Mark McMahon \\ Martin Kotuliak, Jakub Polak}
\date{\today}

\maketitle

\section*{Task 1}

A multiple linear regression model has been estimated to study the relation- ship between $Y =$ violent crime rate (per 100,000 people), $X_1 =$ poverty rate (percentage with income below the poverty line) and $X_2 =$ percentage living in urban area. Data are collected in 51 cities in the U.S.\\

The relevant equations that relate estimate, standard error, T-statistic, R-squared and Sum of Squares of residuals, regression and total are folloqwing.

\[ \frac{\beta_i}{se(\beta_i)} = T_i \qquad R^2 = \frac{\SSreg}{\SST}= 1-\frac{\SSR}{\SST}  \qquad \SST = \SSreg + \SSR
\]

Using these we just plug in the corresponding information that is already provided and compute the missing values.

\begin{knitrout}
\definecolor{shadecolor}{rgb}{0.969, 0.969, 0.969}\color{fgcolor}\begin{kframe}
\begin{alltt}
\hlstd{a} \hlkwb{<-} \hlopt{-}\hlnum{498.683} \hlopt{/} \hlnum{140.988}
\hlstd{b} \hlkwb{<-} \hlnum{4.885} \hlopt{*} \hlnum{6.677}
\hlstd{c} \hlkwb{<-} \hlnum{9.112} \hlopt{/} \hlnum{6.900}
\hlstd{d} \hlkwb{<-} \hlnum{1841257.15} \hlopt{/} \hlstd{(}\hlnum{1} \hlopt{-} \hlnum{0.5708}\hlstd{)}
\hlstd{e} \hlkwb{<-} \hlstd{d} \hlopt{-} \hlnum{1841257.15}
\end{alltt}
\end{kframe}
\end{knitrout}

The table below reports the output with filled in missing information.

\begin{center}
\begin{tabular}{ l r r r r }
 \hline
            & Est.            & s.e.              & t-value         & p-value \\
 \hline
 Intercept  & -498.683  & 140.988 & $^{a\;}$-3.537 & 0.009 \\
 $X_1$      & $^{b\;}$32.617 & 6.677  & 4.885     & 0.001 \\
 $X_2$      & 9.112  & $^{c\;}$1.321 & 6.900      & 0.001 \\
 \hline
 $R^2$      & 0.5708            & & & \\
 SSreg      & $^{e\;}$2448718 & & & \\
 SSR        & 1841257.15        & & & \\
 SSTotal    & $^{d\;}$4289975 & & & \\
 \hline
\end{tabular}
\end{center}

The the coefficient of determination $R^2$ is $0.5708$. This value measures the proportion of the variance in $Y$ explained by the model. Hence, 57.08 \% of the sample variability of $Y$ can be explained by the linear combination of $X_i$'S given the sample data.\\

To compute the overall F-test we use the equation below and the statistic then follows an F distribution with corresponding degrees of freedom.
\[ F = \frac{\SSreg/p}{\SSR/(n-(p+1))} \sim F_{n,n-(p+1)}
\]

The data are collected in 51 cities, so $n=51$ and we have 2 predictors, so $p=2$. Other values we can easily obtain from the filled table above.
\begin{knitrout}
\definecolor{shadecolor}{rgb}{0.969, 0.969, 0.969}\color{fgcolor}\begin{kframe}
\begin{alltt}
\hlstd{(f} \hlkwb{=} \hlstd{(e} \hlopt{/} \hlnum{2}\hlstd{)} \hlopt{/} \hlstd{(}\hlnum{181257.15} \hlopt{/} \hlstd{(}\hlnum{51}\hlopt{-}\hlstd{(}\hlnum{2}\hlopt{+}\hlnum{1}\hlstd{)) ))}
\end{alltt}
\begin{verbatim}
## [1] 324.2312
\end{verbatim}
\end{kframe}
\end{knitrout}

Hence the F-statistic has a value of 324.2311914. To interpret this, the global F-test, tests a null hypothesis that all regression coefficients are simultaneously 0. In a mathematical notation $H_0 : \beta_1 = \beta_2 = 0$. To evaluate the test, we can compute its p-value. It is a quantile of the corresponding F-distribution for the given statistic or mass under the distribution.

\begin{knitrout}
\definecolor{shadecolor}{rgb}{0.969, 0.969, 0.969}\color{fgcolor}\begin{kframe}
\begin{alltt}
\hlstd{(p} \hlkwb{=} \hlkwd{pf}\hlstd{(f,}\hlnum{2}\hlstd{,}\hlnum{48}\hlstd{,}\hlkwc{lower.tail} \hlstd{=} \hlnum{FALSE}\hlstd{))}
\end{alltt}
\begin{verbatim}
## [1] 1.319003e-28
\end{verbatim}
\end{kframe}
\end{knitrout}

\section*{Task 2}

\section*{Task Template}
You can test if \textbf{knitr} works with this minimal demo. OK, let's
get started with some boring random numbers:

\begin{knitrout}
\definecolor{shadecolor}{rgb}{0.969, 0.969, 0.969}\color{fgcolor}\begin{kframe}
\begin{alltt}
\hlkwd{set.seed}\hlstd{(}\hlnum{1121}\hlstd{)}
\hlstd{(x}\hlkwb{=}\hlkwd{rnorm}\hlstd{(}\hlnum{20}\hlstd{))}
\end{alltt}
\begin{verbatim}
##  [1]  0.1449583  0.4383221  0.1531912  1.0849426  1.9995449 -0.8118832  0.1602680
##  [8]  0.5858923  0.3600880 -0.0253084  0.1508809  0.1100824  1.3596812 -0.3269946
## [15] -0.7163819  1.8097690  0.5084011 -0.5274603  0.1327188 -0.1559430
\end{verbatim}
\begin{alltt}
\hlkwd{mean}\hlstd{(x);}\hlkwd{var}\hlstd{(x)}
\end{alltt}
\begin{verbatim}
## [1] 0.3217385
## [1] 0.5714534
\end{verbatim}
\end{kframe}
\end{knitrout}

The first element of \texttt{x} is 0.1449583. Boring boxplots
and histograms recorded by the PDF device:

\begin{knitrout}
\definecolor{shadecolor}{rgb}{0.969, 0.969, 0.969}\color{fgcolor}\begin{kframe}
\begin{alltt}
\hlkwd{par}\hlstd{(}\hlkwc{mar}\hlstd{=}\hlkwd{c}\hlstd{(}\hlnum{4}\hlstd{,}\hlnum{4}\hlstd{,}\hlnum{.1}\hlstd{,}\hlnum{.1}\hlstd{),}\hlkwc{cex.lab}\hlstd{=}\hlnum{.95}\hlstd{,}\hlkwc{cex.axis}\hlstd{=}\hlnum{.9}\hlstd{,}\hlkwc{mgp}\hlstd{=}\hlkwd{c}\hlstd{(}\hlnum{2}\hlstd{,}\hlnum{.7}\hlstd{,}\hlnum{0}\hlstd{),}\hlkwc{tcl}\hlstd{=}\hlopt{-}\hlnum{.3}\hlstd{,}\hlkwc{las}\hlstd{=}\hlnum{1}\hlstd{)}
\hlkwd{boxplot}\hlstd{(x)}
\hlkwd{hist}\hlstd{(x,}\hlkwc{main}\hlstd{=}\hlstr{''}\hlstd{)}
\end{alltt}
\end{kframe}

{\centering \includegraphics[width=.4\linewidth]{figures/plot-boring-plots-1} 
\includegraphics[width=.4\linewidth]{figures/plot-boring-plots-2} 

}



\end{knitrout}

Do the above chunks work? You should be able to compile the \TeX{}
The first element of \texttt{x} is 0.1449583. Boring boxplots
and histograms recorded by the PDF device:

\begin{knitrout}
\definecolor{shadecolor}{rgb}{0.969, 0.969, 0.969}\color{fgcolor}

{\centering \includegraphics[width=.8\linewidth]{figures/plot-boring-plots2-1} 

}



\end{knitrout}

Do the above chunks work? You should be able to compile the \TeX{}

\end{document}
